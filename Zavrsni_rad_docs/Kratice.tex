%%%%%%%%%%%%%%%%%%%%%%%%%%%%%%%%%%%%%%%%%%%%%%%%%%%%%%%%%%%%%%%%%%%%%%%%%%%%%%%%%%%%%%%%%%%%%%%%%
%% ovo je jednostavniji (ali neautomatski) primjer definiranja liste akronima, a student neka to zamijeni svojim kraticama i doda sve koje zeli
%% ako se ne zeli deklarirati popis kratica, onda ga staviti pod komentar u glavnoj datoteci
%
%   \item[{\bf HTML}]	Hypertext Markup Language
%   \item[{\bf AJAX}]	Asynchronous JavaScript and XML

%%%%%%%%%%%%%%%%%%%%%%%%%%%%%%%%%%%%%%%%%%%%%%%%%%%%%%%%%%%%%%%%%%%%%%%%%%%%%%%%%%%%%%%%%%

%%%%%%%%%%%%%%%%%%%%%%%%%%%%%%%%%%%%%%%%%%%%%%%%%%%%%%%%%%%%%%%%%%%%%%%%%%%%%%%%%%%%%%%%%%
% Sofisticiraniji nacin definiranja i uporabe kratica je preko sljedeće sintakse

 \addcontentsline{toc}{chapter}{Pojmovnik}
% primjeri definicije kratica
% ove pojmove zamijenite nekim svojima i po tom predlosku nadogradite listu po potrebi
\newacronym{nfc}{NFC}{Near Field Communication}
\newacronym{wlan}{WLAN}{Wireless Local Area Network} 
\newacronym{gsm}{GSM}{Global System for Mobile (Communications)}


%::::::::: UPORABA KRATICA U TEKSTU :::::::::::::::::
% u tekstu jednostavno na mjestu gdje želite koristiti određenu kraticu, upotrijebite naredbu 
%  \gls{id_kratice}, kao npr. \gls{nfc}
% i u tekstu će vam automatski biti ubačena kratica kako je definirana u drugoj zagradi u gornjim definicijama, a ako je u uporabi prvi puta, tada će prvo biti naveden puni naziv, kako je definiran u trećoj zagradi, a potom kratica. Za sve ostale slučajeve uporabe, bit će navedena samo kratica.

%:::::::::::::::::::::::::::::::::::::::::::::::::::::
%  podsjetnik nekoliko mogućih oblika sintakse
% opća uporaba: \gls{nfc} % može i za rječnik i za kratice. Prvi poziv daje dugi i kratki naziv (redoslijedom koji je specificiran u preambuli pomocu \setacronymstyle, a od drugi puta nadalje samo kraticu.
% ako želite nametnuti baš neki oblik korištenja kratice, imate sljedeće naredbe
%\acrshort{nfc} \\  % samo akronim
%\acrfull{nfc} \\   % akronim i puni naziv
%\acrlong{nfc} \\   % samo puni naziv

%::::::::::::::::::::::::::::::::::::::::::::::::::::
% Kako se generira Pojmovnik u tekstu:
%1. pokrenite LaTeX kompilaciju 1x
%2. u Command Promptu odite radnu mapu gdje su vam datoteke diplomskog rada i utipkajte
%	makeindex -s myDoc.ist -o myDoc.gls   myDoc.glo
%	
%	gdje myDoc zamijenite imenom svoje glavne .tex datoteke (JMBAG_Ime_Prezime.tex)
%3. pokrenite LaTeX kompilaciju jos jednom

%::::::::::::::::::::::::::::::::::::::::::::::::::::