\chapter{Uvod}

Kada pričamo o detekciji sudara u računalnoj grafici, zapravo pričamo o algoritmima koji nam omogućuju brzu detekciju sudara između objekata na sceni. Detekcija sudara temelji se u konačnosti na preklapanju primitiva(trokut, kvadrat, poligon,...)\cite{1}, tj. točaka primitiva objekata. U praksi danas postoji cijeli niz algoritama koji omogućuju brzu detekciju sudara između objekata \cite{1}. Oni se mogu podijeliti na 2 dijela:
\begin{itemize}
	\item Algoritme na podjelu objekata
	\item Algoritme na podjelu prostora
\end{itemize}
Postoje i još neki drugi algoritmi, no u ovom radu usredotočiti ćemo se na ova 2 najčešća pristupa. Ovisno o našim potrebama, odabiremo algoritam koji nam najviše odgovara. U ovom radu bit će opisana oba pristupa i prednosti i mane svakoga.

Prilikom sudara, objekti imaju i neku fizikalnu reakciju na sudar. Ovisno o tome, događa se i gubitak energije na objektima. U ovome radu biti će posvećena posebna pažnja u gubitku energije gibanja prilikom sudara neovisno o tome događa li se sudar od zida, ili od drugog objekta. U početku je zamišljeno da objekti u ovome radu budu kuglice s obzirom na jednostavnu implementaciju i vjerni prikaz detekcije sudara. Na objekte također mora djelovati neka vanjska sila, objekti se ne mogu kretati u prostoru sami od sebe. Odlučeno je da će na cijeloj sceni djelovati gravitacijska sila koja vuče objekte prema "podlozi" od koje se zatim objekti odbijaju. Kako je i već spomenuto, objekti prilikom odbijanja gube energiju.

Smisao ovog rada bio je vjerno prikazati i usporediti efikasnost algoritma u odnosu na primitivnu detekciju sudara gdje se provjerava sudar jednog objekta u odnosu na sve ostale, itd. . U konačnosti, kada se efikasnost algoritma svede do zadovoljavajuće razine, objekti dobivaju neka svojstva da animacije dobiju smisao i dodatnu ljepotu (npr. sudari kuglica na biljarskom stolu, kapljice vode, ...).


%%%%  POGLAVLJE ZAVRSENO  %%%%%
